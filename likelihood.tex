\documentclass{article}
\begin{document}

\newcommand\Nrm{\mathcal{N}}

\newcommand\Ideal{\gamma}
\newcommand\FallOff{\rho}
\newcommand\NoiseVar{\Sigma}
\newcommand\Mix{\alpha}

\newcommand\Feature{f}
\newcommand\Features{\mathbf{f}}
\newcommand\Model{z}
\newcommand\Params{\theta}

\newcommand\UnderlyingF{\mu_k}

The image contains $n \times m$ pixels.

There are $k$ features $\Feature_k$ observed at each pixels.

A reconstruction $\Model=\{\Model_i\}$ consists of a $y$--coordinate
for each image column $i$. So each $(i,\Model_i)$ is the coordinates
of a pixel on the floor/wall intersection.

Hyper--parameters $\Params$ consist of (for each of $k$ features):
\begin{itemize}
  \item{$\Ideal_k\in\Re$, the ``ideal'' value of the feature $\Feature_k$ we
    expect to observe at the floor/wall intersection. This
    is the same for all columns.}
  \item{$\FallOff_k>0$, the ``fall-off'' rate at which $\Feature_k$
    diminishes as we move away from the floor/wall intersection
    $(i,z_i)$ (is actually the variance of an un--normalized
    Gaussian).}
  \item{$\NoiseVar_k > 0$, the Gaussian variance for the noise model in
    feature $\Feature_k$}
  \item{$\alpha_k \in [0,1]$, the probability that the measurement $\Feature_k$
    arose from the signal we're interested in as opposed to a
    background noise process.}
\end{itemize}

In addition, $\Params$ contains the mean $\mu_0$ and variance $\Sigma_0$ for the
background noise process, which is shared by all $k$ features.

A feature vector $\Features$ at row $i$, column $j$ is then
distributed as
\begin{equation}
  P(\Features ~|~ \Model, \Params) = \prod_k
    \Mix_k \Nrm(\Feature_k; \UnderlyingF, \NoiseVar_k) +
    (1-\Mix_k) \Nrm(\Feature_k; \mu_0, \Sigma_0)
\end{equation}
where
\begin{equation}
  \UnderlyingF = \Ideal_k \Nrm(j; \Model_i, \rho_k)
\end{equation}

\end{document}
